
\documentclass[10pt,  english, makeidx, a4paper, titlepage, oneside]{book}
\usepackage{babel}
\usepackage{fancyhdr}
\usepackage{makeidx}
\usepackage{titlesec}
\usepackage{listings}
\usepackage{booktabs}
\usepackage{hyperref}
\usepackage{circuitikz}

\newenvironment{listato}{\footnotesize}{\normalsize }

%\pagestyle{empty}

\textwidth 15.5cm
\textheight 23cm
\topmargin -1cm
\oddsidemargin 0cm
\linespread{1.1}

\pagestyle{fancy}
\lhead{}
\chead{Microelectronic Systems}
\lfoot{}
\cfoot{}
\rfoot{}
\rhead{\thepage}

\usepackage{graphicx}
\usepackage{amsmath}
\usepackage{amsfonts}
\usepackage{amsthm}
\usepackage{amssymb}
%\oddsidemargin -1.1cm
\usepackage{graphicx}
\usepackage{caption}
\usepackage{float}
\usepackage{amsmath}
\usepackage{amssymb}
\usepackage{amsfonts}
\usepackage{amsthm}
%\usepackage{subscript}
\usepackage{empheq}
\usepackage{verbatim}
\usepackage{fancyvrb}
\usepackage{multicol}

% Definisce i colori per il codice in VHDL

\lstdefinestyle{vhdl}{
   language=vhdl,
   frame=single,
   basicstyle=\scriptsize,
   breaklines=true,
   captionpos=b,
   keepspaces=true,
   backgroundcolor=\color{backcolor},
   keywordstyle=[1]\color{blue}\bf,
   keywordstyle=[2]\color{red}\bf,
   keywordstyle=[3]\color{cyan!50}\bf,
   stringstyle=\color{orange},
   commentstyle=\color{comment},
   tabsize=2,
    numbers=left,
   showspaces=false,
   showstringspaces=false,
   showtabs=false,
   moredelim=[s][\textcolor{green}]{component}{is},
   morekeywords=[1]{
      library, use ,all,entty,is,port,in,out,end,architecture,of, body,
      function, variable, begin,and,or,Not,downto,ALL, signal, process, if,
      else, elsif, case, when, then, range, to, component, type, with, select,
      others, constant, inout, buffer, map, true, false, array, subtype, wait,
      wait for, generic, =, <, >, <=, >=, =>,
   },
   alsoletter={=, <, >},
   morekeywords=[2]{
          STD_LOGIC_VECTOR,STD_LOGIC,IEEE,STD_LOGIC_1164, work, local, real,
          math_real, time, NUMERIC_STD,STD_LOGIC_ARITH,STD_LOGIC_UNSIGNED,
          std_logic_vectr, std_logic, ieee, numeric_std, std_ulogic,
          std_logic_1164, natural, bit, bit_vector, signed, unsigned,
          boolean, integer
    },
    morekeywords=[3]{rising_edge, falling_edge, resize, to_signed, to_unsigned},
    morecomment=[l]{--},
    morecomment=[s][\color{orange}]{'}{'},
    numbers=left,
}


\titleformat{\chapter}[display]
{\normalfont\Large\filcenter\sffamily}
{\titlerule[0.5pt]%
\vspace{1pt}
\titlerule
\vspace{1pc}
\LARGE\MakeUppercase{\chaptertitlename} \thechapter
}
{1pc}
{\titlerule
\vspace{1pc}
\Huge}

\makeindex

\begin{document}

\frontmatter
\begin{titlepage}
\vspace{0cm}
\centerline{
\includegraphics[width=9cm]{./logopoli}} 
\vspace{0.5cm}
\vspace{2.5cm}
\centerline{\huge\sf Microelectronic Systems}
\vspace{1cm}
\centerline{\Huge\sf DLX Microprocessor: Design \& Development}
\bigskip
\centerline{\huge\sf Final Project Report}
\vspace{1cm}
\centerline{\Large Master degree in Computer Engineering}
\bigskip
\centerline{\Large Master degree in Electronics Engineering}
\vspace{4.5cm}
%%%%%%%%%%%%%%%%%%%%%%%%%%%%%%%%%%%%%%%%%%%%%%%%%%%%%%
%
\centerline{\large Referents: Prof. Mariagrazia Graziano, Giovanna Turvani}
\bigskip
\vspace{1cm}
\centerline{\large Authors: group\_16}
\bigskip
\centerline{\large \textbf{Battilana Matteo, La Greca Salvatore Gabriele, Pollo Giovanni}}
%
%%%%%%%%%%%%%%%%%%%%%%%%%%%%%%%%%%%%%%%%%%%%%%%%%%%%%%
\vspace{2cm}
\centerline{\large \today}
\end{titlepage}


\shipout\null

\chapter{Feature}
\begin{center}

\vfill
\begin{tabular}{ l c }
	\textbf{Frequency} & 400 MHz \\ 
	\textbf{Slack} & 0 (MET) \\  
	\textbf{Area} & 35278.24 \\  
	\textbf{Total Dynamic power} & 9.88 mW \\  
	\textbf{Cell Leakage power} & 653.22 $\mu$W
\end{tabular}

\end{center}
\hfill
\\[60pt]
\hfill

\begin{multicols}{2}
\begin{itemize}
\item 32 bit RISC CPU
\item 400 MHz operating frequency
\item Up to 4 GB of Addressable Memories
\item Five stages pipeline
\item Full Hazard Detection \& Control
\item System Tick Timer
\item Enhanced T2 3-level Shifter
\item Hardwired Control Unit
\end{itemize}

\columnbreak

\begin{itemize}
\item Expanded instruction set
\item Fast Subroutine call with Windowing RF
\item Word, Half and Byte addressable Memory
\item Signed/Unsigned multipurpose comparator
\item Advanced Datapath with reduced delay
\item Efficient CLA-Carry Select Adders
\item Enhanced Booth's Multiplier
\end{itemize}
\end{multicols}
\vfill

\tableofcontents
\lstlistoflistings

\mainmatter

\chapter{Introduction}

\section{Abstract}
The goal of this project is to build from scratch a working implementation of a DLX. In order to achieve the goal, some known blocks, created during laboratories, were used. 

The second step was the design of the datapath, done in the best possible way to obtain a high optimization and performance level. Some optimization examples that will be explained more in depth in the document are the use of the P4 adder and the Booth multiplier inside the ALU, the comparator and many others. 

The third step was the design of the control unit. The choice fell on the microprogrammed version that guaranteed the pipeline implementation. In order to simplify the maintainability of the control unit, some struct-like constructs were used in the VHDL code. 

In the fourth step, a very exhaustive testing has been executed. All the proposed asm codes were verified, but some well-known algorithm (bubble sort, Fibonacci and factorial) were written and tested.

Last but not least, the DLX has been synthetized using synopsis, and a post-synthesiss simulation has been executed. 

Thanks to the datapath optimization and the synthesis optimization, the microprocessor proposed in this paper reached a peak speed of 400MHz.  
\section{Workflow}
As many tools we used to automate the working process, all of them are explained in this section.

The first and most important tool was versioning control. The choice fell on GitHub. Thanks to this, the team managed all versions of the code and stepped back if any problem occurs. In addition to that, team communication and issue management were very straightforward, thanks to the possibilities offered by the tool. Another handy feature was the milestone, which allows the team to be on time and respect deadlines. 

The used programming technique was the pair programming, that allows to write code and checking its correctness at the same time. This technique was particularly useful in difficult part of the projects. To exploit pair programming the extension used was Live Share for Visual Studio Code (\url{https://github.com/MicrosoftDocs/live-share}). Indeed, for the easier steps, the use of the branches and pull request gave the possibility of parallel working and drastically reduced the presence of conflicts. 

The last thing to point out was the intensive use of scripting for compiling VHDL code, simulating it, adding wave and then synthetizing the full project. All this scripts are reported in the appendix.
\tikzset{
% async
latch/.style={flipflop, flipflop def={t1=D, t6=Q, t3=CLK, 
t4=\ctikztextnot{Q}}},
flipflop SR/.style={flipflop, flipflop def={t1=S, t3=R, t6=Q, 
t4=\ctikztextnot{Q}}},
% sync
flipflop D/.style={flipflop, flipflop def={t1=D, t6=Q, c3=1, 
t4=\ctikztextnot{Q}}},
 flipflop T/.style={flipflop, flipflop def={t1=T, t6=Q, c3=1, 
t4=\ctikztextnot{Q}}},
flipflop JK/.style={flipflop,
flipflop def={t1=J, t3=K, c2=1, t6=Q, t4=\ctikztextnot{Q}}},
% additional features
add async SR/.style={flipflop def={%
tu={\ctikztextnot{SET}}, td={\ctikztextnot{RST}}}},
dot on notQ/.style={flipflop def={t4={Q}, n4=1}},
}


\chapter{Hardware Architecture}

\section{Overview}

\begin{figure}[ht]
    \centering
    \includegraphics[width=0.9\textwidth]{chapters/2_dlx/images/DLX.png}
    \caption{Schematic of the DLX}
    \label{DLX}
\end{figure} 

\section{Pipeline Stages}
\section{Control Unit}
\section{Memories Interface}
\section{Instruction Set}
\chapter{Fetch Stage}

\section{Instruction Register}
\section{Program Counter}
\section{Jump and Branch Management}
\chapter{Decode Stage}

\section{Instruction Decode}
\section{Register File and Windowing}
The general structure of a register file is based on a decoder that takes the selection input (so the address of the desired register) and enables it (using also the enable signal). At this point, an input signal will contain the value to be written. On the other hand, a read signal is used to select among all the registers.\newline\newline
The DLX presented in this document has been enhanced in order to be able to manage subroutine in a transparent manner from the point of view of the user. For this reason, the DLX must be able to handle subroutines, and so the context switching, that consists in saving the
registers content in order to be restored once the procedure has been completed. The straightforward solution is to save into the memory all registers but this is not feasible in terms of delay, since for 32 registers we will need 32 clock cycles; if you image this in a pipeline, this corresponds to a long stall each time a procedure is called.

A windowed register allows to reduce the overhead due to the context switch; the basic idea
is to split the available registers in the physical register file into blocks, called \textit{windows}. We have limited amount of physical registers in the register file, for this reason a finite number of windows are defined. Each window is assigned to a subroutine, so that the procedure can write only on those register. This is transparent from the point of view of the CPU, that sees all registers available. Thus, the physical register file has a wrapper around it with a logic and a Register Management Logic (MML) that allows to perform the translation between the CPU requests to corresponding window for the running procedure.\newline\newline
What if the number of called procedure is larger than the number of available windows? The main
memory is involved only when there are no free windows in the register file. In this case, the oldest allocated window is swapped into the main memory, so that the new one can be allocated. Obviously, once all the recursion chain has been unrolled, the swapped window in the memory must be restored into the register file.
All windows, so each procedure, has 4 blocks of 8 registers each one:
\begin{enumerate}
	\itemsep0sp
	\item \textbf{IN}: the first block is dedicated to the data inherited from the parent routine (OUT);
	\item \textbf{LOCALS}: contains the registers that are dedicated to the procedure;
	\item \textbf{OUT}: is dedicated to the variables to be passed to the child routine, that is the IN of the next sub-procedure
	\item \textbf{GLOBAL}: the last block is common to every windows.
\end{enumerate}
When a procedure is called, the first LOCALS and OUT blocks are allocated from the physical file register and assigned to it (because IN is the OUT of the previous one).

By calling many nested procedures, at some point there will be no free windows; for this reason the oldest is de-allocated from the physical FR and swapped to the main memory, the operation is called \textbf{SPILL}. This it accomplished by using a support pointer, called \textbf{Saved Window Pointer SWP} that stores the point of the spilled data, exactly the end of the LOCALS block (only IN and LOCAL are spilled, the OUT block is not spilled because is the IN of the next sub-procedure). In practice it define the position of the last free cell. Notice that this operation cannot be executed in one clock cycle: each register is spilled once at a clock cycle.\newline\newline
On the other hand, when the last procedure in the chain is finished, the other are unrolled; if some of them have been spilled, a \textbf{FILL} must be executed before the actual execution. This can be achieved by, firstly decrement CWP by 16 and check if now CWP $>>$ SWP.\newline\newline
It's important to notice that the implementation of the entire register file has been implemented in Structural. Is is composed by several components:
\begin{itemize}
	\item \textbf{Decoder}: it is used to generate a single enable signal from a signal on \textbf{NBIT\_ADD} bits; in this way, a register is selected in order to perform a write. The register will check also if a write is requested;
	\item \textbf{Connection matrix}: this block allows to ``highlight" the active windows, the block IN, LOCAL and OUT will be the default destination when writing and reading;
	\item \textbf{Register file}: this block corresponds to the physical registers, composed by rows of Flip-Flops;
	\item \textbf{Select block}: this block is used for the reading, is connected to all the registers and selects, using the read address, the single register to be read;
	\item \textbf{Address generator}: this block is used only when perform a FILL or a SPILL, it generates the address for the registers to be moved from/to the memory. The memory, in this case works exactly like a stack.
\end{itemize}
Additional, but less complex components, have been used in order to manage the management of the CWP and SWP.


\subsection{Decoder}
This block receives as input the \emph{write address} on \textbf{NBIT\_ADD} bits and outputs \(\mathbf{2^{NBIT\_ADD} - 1} \) bits. It has the utility of converting the address of the register at which we need to write into its enable signal. 

The idea is that if the input is \(0b00010\) the output will be \(0b00000000000000000000000000000100\). In fact if the input is decimal 2, it means that we need to write the second register of the \emph{GLOBAL} block. In terms of enable it can be translated by having the bit with index 2 at one. In fact in the output we see that the it with index 2 has value 1, while the others are all 0. 

The output is divided (in the schematic) in order to represent the group of bits. In particular we have that: 
\begin{itemize}
    \item M - 1 DOWNTO 0: bits associated to the \emph{GLOBAL} register
    \item M + N - 1 DOWNTO M: bits associated to the \emph{IN} register
    \item M + 2N - 1 DOWNTO M + N: bits associated to the \emph{LOCAL} register
    \item M + 3N - 1 DOWNTO M + 2N: bits associated to the \emph{OUT} register
\end{itemize}

On the top of the schematic (\autoref{decoder}) we can see an AND logic port between \emph{ENABLE} and \emph{WR} signals. If both \emph{ENABLE} and \emph{WR} are 1, it means that our register need to work. In fact, the output of the dedocer is anded with 1 and so we maintain the value. Otherwise, if one signal between \emph{ENABLE} and \emph{WR} is 0, the output will be 0 and so the AND with the output of the \emph{decoder} will return all 0. 

This signal goes into the \emph{connection matrix}, which is the next block described. 

%% MAKE FONT BIGGER
\begin{figure}[ht]
    \centering
    \includegraphics[width=0.7\textwidth]{chapters/4_DecodeStage/images/Decoder.pdf}
    \caption{Schematic of the Decoder}
    \label{decoder}
\end{figure}
%% MAKE FONT BIGGER

\subsection{Connection Matrix}

With the previous block, we generated all our enable signals. The problem is that we have more windows. So how do we decide which window needs to be activated? Here comes the connection matrix. This block receives as inputs the signal coming from the decoder, the current window, the saved window and the address for the pop (fill) operation. The output is a signal that contains the enable signals ready for all the registers of all windows. 

We have a specific structure for each block:
\begin{itemize}
  \item GLOBAL: the global is the simplest, because it is connected directly to the output
  \item IN: for this block we AND the IN bits coming from the decoder with the bit (that is extended) of the related window. For example if we are evaluating the IN of the first window, we will AND the IN bits with the bit 0 of the current window.
  \item OUT: for this block we AND the OUT bits coming from the decoder with the bit (that is extended) of the previous related window. For example if we are evaluating the OUT of the first window, we will AND the OUT bits with the bit 4 of the current window (supposing our window has 5 bits).
  \item LOCAL: for this block we AND the LOCAL bits coming from the decoder with the bit (that is extended) of the related window. For example if we are evaluating the LOCAL of the first window, we will AND the LOCAL bits with the bit 0 of the current window.
\end{itemize}

For the IN and OUT we then an OR between the two outputs (the logic can be seen in the schematic), while for the LOCAL we don't have anything.

In addition to that, the connection matrix also manages the saved window, used for the pop (fill) operation. First we need to invert the addr\_pop, because when we execute the pop operation, we restore data starting from the last one (we are using a STACK).
The addr\_pop\_inverted is composed like this:
\begin{itemize}
  \item 2N - 1 DOWNTO 0: we have the IN bits 
  \item N - 1 DOWNTO 0: we have the LOCAL bits
\end{itemize}

The signal is splitted into two wires and is anded with the saved related saved window pointer. 

In the end, we definitely OR the output of the previously described OR with the output of this AND. This is visible in the schematic. 


\newpage

\subsection{Register File}

The next block is the Register File, that is a sequence of registers. The important thing to notice in our design is how we managed the data that goes into the registers. We have two choice, data\_in and from\_mem. In order to choose we decided to use multiplexers. 
We have a multiplexer for each window. The signal used to drive the multiplexer is the saved window pointer, rotated right by 1 position and anded with the pop signal. In fact, we select from\_mem only when the pop signal is 1, otherwise we need to select data\_in. 
We use the saved window pointer shifted by 1 because....... TODO

\section{Hazard Control}
\section{Comparator}
- Unsigned things 

\section{Jump and Branch decision}
\section{Next Program Counter computation}
\chapter{Execute Stage}

\section{ALU: Arithmetic Logic Unit}
\subsection{Adder}
\subsection{Multiplier}
\subsection{Logic Operands}
\subsection{Shifting}

\section{Set-Like Operations unit}
- setcmp
\chapter{Memory Stage}

\section{Load-Store Unit}
- Unsigned things
\section{Address Mask Unit}
\chapter{Write Back Stage}

Mux selects from Memory Output (LoadStore Unit) or ALU output.

Signal to enable register file write.
Registers to delay the write register address
\chapter{Testing and Verification}
Testing a verification is used to verify the correctness of parts or the whole DLX. Testing can be defined as the process of observing and verifying the results of a component under specific conditions in order to find possible inconsistency between the actual and the required behaviour.

The flow used to verify a component follows multiple steps:
\begin{itemize}
	\itemsep0sp
	\item Creation of the Testbench
	\item Simulation
	\item Post Synthesis simulation (after the Synthesis and Optimization, refer to section \ref{sec:syn_opt})
\end{itemize}

\section{Testbenches}
All Testbenches developed for the DLX verification have been implemented using VHDL. The most critical components have been tested with their own testbenches, like:
\begin{itemize}
	\itemsep0sp
	\item Booth's multiplier
	\item P4 adder
	\item Comparator
	\item ALU
	\item Shifter
	\item Windowing Register File
\end{itemize}
While the others smaller components have been indirectly intensively tested while verifying the correct behaviour of the DLX itself.\newline\newline
The DLX Testbench is implicitly based on two processes, the first one manages and creates only the clock signal with a period of 1 ns, while the second one asserts the reset signal for the first clock period and the negates it in order to start the correct test execution.

In order to create the most flexible test possible, multiple components have been instantiated; these are providing all the external interfaces the DLX needs. So, besides the DLX component, the Testbench contains also the following component:
\begin{itemize} 
	\item Instruction RAM (IRAM): this is a read only memory that takes an external file, in this case a .mem file, and uses it as source for fetching the instruction. When the reset is performed, all the instructions in the file are written in an internal array of 32 bits. At each clock cycle, the array is indexed with an index that comes from the outside, that corresponds to the DLX \texttt{PC}.
	\item Data RAM (DRAM): in a similar way with the respect to the IRAM, the DRAM is loaded at the startup using the same .mem explained before. In fact, this compiled file does not contains only the code instruction but, after a blank line, also all the memory content the program should have once it starts. Since a DRAM must be also writable, the memory is implemented as a read/write one. 
	
	In order to simplify the verification of the memory content, each time a value is written and external file, that is the exact copy of the DRAM content, is updated too.
	In order to correctly access the data that are stored into the memory, the address comes from outside and corresponds to the one used in the Memory Stage (refer to \ref{chp:memory_stage}).
	
	To replicate a more realistic environment and increase the test reliability, the DRAM has been enhanced with a \texttt{ready} signal that is `1' when it is ready. This has been implemented with a counter, that only after \texttt{data\_delay} clock cycles make the memory ready and allows to generate a meaningful value.
	
	Last but not least, the DRAM must be able to correctly manage all the different data sizes both during the write and the read operation. So, an additional signal on two bits, called \texttt{MAS}, has been included. A more accurate description of the memory data size management is available at section \ref{mas}.
\end{itemize}

\section{Simulation}
\section{Post Synthesis Simulation}
\chapter{Physical Design}
The trend nowadays is to build more complex system (in terms of transistors) in less time (reduce the time-to-market), so there is the need of some powerful tools that allows having optimized ICs. The design flow strategy is based on multi-abstraction 3-step iteration:
\begin{enumerate}
    \item The hardware is described using a Hardware Description Language, as VHDL
    \item The Synthesis phase takes as input the abstract model and generates a more detailed model that contains additional information about timing, power consumption and area. The next step is the Optimization one, which is used in order to generate an equivalent behaviour circuit and at the same time satisfy some conditions, like timing
    \item A post-synthesis simulation is run to check the functional properties of the final model
\end{enumerate}
\section{Synthesis}
\label{sec:syn_opt}
The synthesis has been done with intensive script usage; in fact, two scripts have been developed in order to set up the environment, perform the synthesis and clean up all the useless temporary files generated during the process.\newline\newline
The first script is a bash script (refer to Appendix \ref{bash_syn}) and, as anticipated before, it is used to set up the environment by coping the \texttt{.synopsys\_dc.setup} file, copy the library and call the synthesis script suing \texttt{dc\_shell}.
Once the synthesis is over, the bash script removes all temporary folders like \texttt{ARCH}, \texttt{DOBY}, etc \dots and moves the synthesized DLX, in both Verilog and VHDL, and all the generated reports into a specific folder.\newline\newline

The second script, that is run under the \texttt{dc\_shell} to perform the actual synthesis, executes multiple steps (refer to Appendix \ref{tcl_syn}):
\begin{itemize}
    \itemsep0sp
    \item Analyze all the .vhd files needed for the DLX
    \item Elaborate the DLX design, by correctly configuring the generics
    \item Set the wire model and create a clock, that is the constraint
    \item Perform the compilation
    \item Save the synthesized DLX
    \item Save the timing, area and power report 
\end{itemize}
The clock timing, which is set to 2.5 ns, has been selected after many trials and errors, in order to find the lowest possible value. Critical paths, like the one that passes through the adder, has been reduced using an optimized design, e.g. P4 adder.\newline\newline
All complete reports are detailed in the Appendix \ref{ap3}; from the area report, it's possible to observe that the total cell area is 35278.25. It is divided into 19747.84 for the combinational part and 15530.41 for the non-combinational one.\newline\newline
Given a timing constraint, another important piece of information that can be extracted from the timing report is the \textit{slack}. It represents the time margin that the worst path has; in this way, the clock in the synthesis script can be reduced as much as possible in order to increase performance.

\section{Place and Route}
The Placement step consists of placing all the block and I/O pins within a defined area that is the chip. After this macro-step all the units are placed and so the routing is performed in order to connect all blocks. After the place and route are both completed, a simulation ensures that everything is correct. The final result can be observed in Figure \ref{fig:phy_end}.


\begin{figure}[h!]
    \begin{subfigure}[t]{.45\textwidth}
        \centering
        \includegraphics[width=.9\textwidth]{chapters/9_PhysicalDesign/images/pre_routing.png}
        \caption{DLX processor before routing, only logical connections are present}
        \label{fig:pre_routing}
    \end{subfigure}\hfill
    \begin{subfigure}[t]{.45\textwidth}
        \centering
        \includegraphics[width=.9\textwidth]{chapters/9_PhysicalDesign/images/phy_end.png}
        \caption{DLX processor after Place and Route phases}
        \label{fig:phy_end}
    \end{subfigure}
    \caption{DLX Place and Route}
\end{figure}

In order to obtain a fully placed and routed DLX, many steps are performed:
\begin{itemize}
    \item Structuring the Floorplan: in this step the Verilog file has been loaded using a global file called \texttt{DLX.globals} and a specif amount of internal area is dedicated to the core, while the external one is used for the power rings;
    \item Power distribution: around the core, two metal rings have been located for distributing the power supply, so both GND and $V_{dd}$. This is not enough, since the power and ground signals must be correctly distributed. For this reason, multiple vertical metal wires, called strips, has been added to the physical layout. There is a trade-off in the number of stripes since a high number of them could lead to some problems during the cells routing. Moreover, horizontal wires have been placed to prepare $V_{dd}$ and GND for the standard cells. The results are visible in Figure \ref{stripes};
    \begin{figure}[h]   
        \centering
        \includegraphics[width=0.38\textwidth]{chapters/9_PhysicalDesign/images/pwr_distribution.png}
        \caption{Result after placing GND and $V_{dd}$ rings with vertical and horizontal stripes.}
        \label{stripes}
    \end{figure}
    
    \item I/O placing: at this point, cells and I/O pins can be placed. Before filling the empty spaces with filler cells a Post Clock-Tree-Synthesis (CTS) optimization has been performed. The result is visible in Figure \ref{fig:pre_routing};
    \item Routing: the last step is the routing; logical interconnections have been replaced with physical interconnections between cells, considering the available stripes and metal rings. The design is now complete, but a post routing optimization has been performed in order to respect the required timing constraints.
\end{itemize}

Once the Place and Route step has been done, a timing analysis has been performed using the \textit{Innovus Debug timing} in order to visualize the delay distribution. The paths delay distributions is visible at \ref{fig:innovus_delay}.
\begin{figure}[h]   
    \centering
    \includegraphics[width=0.8\textwidth]{chapters/9_PhysicalDesign/images/innvous_delay.png}
    \caption{Result after placing GND and $V_{dd}$ rings with vertical and horizontal stripes.}
    \label{fig:innovus_delay}
\end{figure}

Last but not least, before ending the place and route process, design analysis and verification has been performed in order to ensure that the connectivity and the design rules are respected.
\chapter{Conclusions}
To sum up, after the Synthesis and Optimization phase, the peak frequency is 400 MHz. The generated reports highlight that the main power contribution is the dynamic one. This is due to the complex logic that has been implemented in the entire DLX. This DLX implementation has been intensively tested using complex assembly programs, starting from the arithmetic instructions to the sub-routine management.\\

The entire development has been focused on a modular approach to simplify the integration of new components and possible improvements. For example, further improvements could be the management of exceptions, a cache for the instruction and a branch target buffer.\\

All team members contributed actively and with a proactive approach to all the challenges this project brought. \\

\appendix
% \input{./Appendix/appendix1.tex}
% and so on
%
%%%%%%%%%%%%%%%%%%%%%%%%%%%%%%%%%%%%%%%%%%%%%%%%%%%%%%

\end{document}