\chapter{Execute Stage}

\section{ALU: Arithmetic Logic Unit}
\subsection{Adder}
The straightforward way to implement an adder is to use the Ripple Carry Adder structure, that is composed of $N-1$ Full Adder and one Half Adder (the first), where $N$ is the number of bits of the two operands. This solution is not optimal from a timing point of view due to the time needed to propagate the carry, that defines the critical path, that is the bottleneck. \newline\newline
Since the sum and the subtraction is one of the most common operation, the DLX includes an adder that is based on a CLA - Carry Look Ahead (Sparse Tree) and a Carry Select Like Adder. The complete structure can be seen at figure \ref{fig:P4}.

\begin{figure}[h]
	\centering
	\includegraphics[width=0.5\textwidth]{chapters/5_ExecuteStage/images/P4.pdf}
	\caption{Booth's multiplier on 32 bits}
	\label{fig:P4}
\end{figure}
As said before, the adder is composed of two blocks:
\begin{itemize} 
	\item \textbf{Carry Select Like Adder}: The main point of the Carry Select Adder is that it doubles the complexity of the adder itself in order to obtain better performances. It is composed by two RCA, in order to perform two sums in parallel.
	
	The idea is to compute both the results, on 4 bits in this case, for both when the carry-in is equal to `0' or `1'. In this way, the results is computed in parallel for all the stages, even if the carry-in is `0' or `1'; then the carry-in is used to mux against the two results (on 4 bits) and the two carry-outs. The carry-out will be used as result selection signal for the next Carry Select unit.
	
	We are paying complexity in order to reduce the addition computation time, in fact by having a carry out that is used as carry in for the next state, there is still propagation but is lower.\newline\newline	
	The DLX implementation, instead of using a straightforward implementation of the Carry Select Adder, it uses a modified version of it. It has been implemented using \textit{CLA - Sparse Tree Carry Generator} and a \textit{Carry Select Like Adder}. The base idea is to use the CLA in order to compute a carry every $n$ bits, then these carries are fed into the sum generator that uses them to compute the results in parallel.
	\begin{figure}[H]
		\centering
		\includegraphics[width=1\textwidth]{chapters/5_ExecuteStage/images/carry_sum.pdf}
		\caption{Carry Select Like Adder block for a 32 bits implementation}
		\label{fig:carry_sum}
	\end{figure}
	\item \textbf{Carry Look Ahead - Sparse Tree}: this block is used to compute the carry out every 4 bits. The idea behind the CLA is to compute several carries simultaneously and to avoid waiting until the correct carry propagates from the stage of the adder in which it has been generated. This is done thanks to the \textit{propagate (P)} (that is 1 if the carry-in is equal to the carry-out) and \textit{generate (G)} (that is one if carry-in is 0 and carry-out is 1).
	\begin{table}[H]
		\begin{center}
			\begin{tabular}{ c c c | c c | c c}
				$a$ & $b$ & $c_{in}$ & $out$ & $c_{out}$ & $p$ & $g$ \\
				\hline
				0 & 0 & 0 & 0 & 0 & 0 & 0\\ 
				0 & 0 & 1 & 1 & 0 & 0 & 0\\ 
				0 & 1 & 0 & 1 & 0 & 1 & 0\\ 
				0 & 1 & 1 & 0 & 1 & 1 & 0\\ 
				1 & 0 & 0 & 1 & 0 & 1 & 0\\ 
				1 & 0 & 1 & 0 & 1 & 1 & 0\\ 
				1 & 1 & 0 & 0 & 1 & 0 & 1\\ 
				1 & 1 & 1 & 1 & 1 & 0 & 1\\ 
			\end{tabular}
			\caption{Computation of propagate and generate bits}
		\end{center}
	\end{table}
	\begin{equation} \label{eq:pandg}
		g = a \oplus b \quad\quad p = a \cdot b
	\end{equation}
	The base idea is to write any $s_i$, that is the i-esim bit of the sum and $c_i$, the carry-out at $i$ index in function of $p$ and $g$.
	We now use $p$ and $g$ to express the same result:
	\begin{align*}
		& s_1 = a \oplus b \oplus c_{in} = p_1 \oplus c_0\\
		& c_1 = a \cdot b + a \cdot c + b \cdot c = a_1 \cdot b_1 + (a_1 + b_1) \cdot c_0 =  g_1 + p_1 \cdot c_0
	\end{align*}
	The crucial point is that it's possible to compute the carry at $i$ position only using the initial carry-in $c_{in}$ and $p$ and $g$ generate in the current and previous blocks. Tree are a family of Carry Look Ahead that differ for the carry-logic. They are based always on \textit{propagate} and \textit{generate}. We have that 	
	\begin{align*}
		& carry = g + p \cdot c_{in}\\
		& G_{i:j} = G_{i:k} + P_{i:k} \cdot G_{k-1:j}\\
		& P_{i:j} = P_{i:k} \cdot P_{k-1:j}
	\end{align*}
	where
	\begin{itemize}
		\itemsep0sp
		\item $i \ge k > j$
		\item $G_{x:x} = g_x$ and $P_{x:x} = p_x$
		\item $g_0 = C_{in}$
	\end{itemize}
	The white and gray blocks in the Sparse Tree block at \ref{fig:pg_network}, that are used in the \texttt{carry\_generator} block, are PG and G blocks.
	Normally two blocks are used, the first \textit{G} generates only $G_{i:j}$ and the other \textit{PG} both $G_{i:j}$ and $P_{i:j}$.
	\begin{figure}[h]
		\centering
		\includegraphics[width=0.4\textwidth]{chapters/5_ExecuteStage/images/PG_and_G.pdf}
		\caption{PG and P block}
		\label{fig:PG_and_G}
	\end{figure}
	The base idea is to combine their outputs and take only the G one as carries.
\end{itemize}
So, Carry Look Ahead - Sparse Tree needs a starting block that generates all the $p$ and $g$ for all the couples of bit using the \ref{eq:pandg} equation. The spars tree and the PG network structure is shown in the figure \ref{fig:pg_network}, in the code this block is called \texttt{prop\_gen\_generic} and is made of \texttt{prop\_gen} block in order to compute $p$ and $g$.
\begin{figure}[H]
	\centering
	\includegraphics[width=0.7\textwidth]{chapters/5_ExecuteStage/images/CLA.pdf}
	\caption{Carry Look Ahead - Carry Generator Block}
	\label{fig:pg_network}
\end{figure}
In order to be able to perform both subtraction and sum, the first block must be modified and must include the logic to manage also the carry-in, that in case of a subtraction it is `1'. This is not enough to perform subtraction, in fact, an additional signal called \texttt{SUB\_SUMN} is needed.
So, the same structure can be used to implement the subtraction, by adding an XOR on each B input with the \texttt{SUB\_SUMN} control signal.

We can say that a subtraction in 2's complement can be implemented as $A + \overline{B} + 1$; in order to implement this in the circuit we need to set the carry-in to `1' (so \texttt{SUB\_SUMN} = `1') in order to add 1 and invert the B input by using the XOR. In fact:
\begin{displaymath}
	\begin{array}{c c|c}
		% |c c|c| means that there are three columns in the table and
		% a vertical bar '|' will be printed on the left and right borders,
		% and between the second and the third columns.
		% The letter 'c' means the value will be centered within the column,
		% letter 'l', left-aligned, and 'r', right-aligned.
		x & y & y \oplus q\\ % Use & to separate the columns
		\hline % Put a horizontal line between the table header and the rest.
		0 & 0 & 0\\
		0 & 1 & 1\\
		1 & 0 & 1\\
		1 & 1 & 0\\
	\end{array}
\end{displaymath}
This solution is good because the PG and G block has the same delay, that is driven by G and since both include it, they are the same. Many paths have the same delay and the load on components is good since an output of a block is connected at maximum to 2 other block. These two factors bring a good \textit{equilibrium} to the entire structure.
\subsection{Multiplier}
In order to overcome the limitation of the array multiplier, this DLX implementation includes a modified version of the Booth's multiplier, since the multiplex for the partial values to be added is only on two bits instead of three. The Booth's algorithm copes with 3 bits at a time, so the number of stages is $N/2$ (this corresponds to the number of the encoders) and this allows to speed up the result computation. 
The Booth's algorithm is the following:
\begin{lstlisting}[frame=none, escapeinside={(*}{*)}]
	i = 0
	P = 0
	while i (*$\leq$*) M - 2 loop
		P = P + Vp( (*$B_{i+1}, B_i, B_{i-1}$*) )
		A = A * 4
		i = i + 2
	end loop
\end{lstlisting}
Where \texttt{P} is the final value of the product and during the algorithm execution it will contain the partial result; \texttt{M} represents the number of bit of the multiplicand, in this case $B$. The algorithm takes as convention that $B_{i-1} = 0$. The \texttt{Vp} is a lookup table (see \ref{mult:lut}), that return the value to add to \texttt{P}, according to the 3 bits selected. The value of $A$ is multiplied by 4.

\begin{table}[H]
	\begin{center}
		\begin{tabular}{ c c c | c}
			0 & 0 & 0 & 0\\ 
			0 & 0 & 1 & +A\\ 
			0 & 1 & 0 & +A\\ 
			0 & 1 & 1 & +2A\\ 
			1 & 0 & 0 & -2A\\ 
			1 & 0 & 1 & -A\\ 
			1 & 1 & 0 & -A\\ 
			1 & 1 & 1 & 0\\ 
		\end{tabular}
		\caption{Booth's LUT}
		\label{mult:lut}
	\end{center}
\end{table}

The Booth's multiplier work with 3 main components, supposing $A$ is the multiplicand and $B$ the multiplier:
\begin{enumerate}
	\item $N/2$ encoders in order to take 3 bits from the operand $B$; the two LSB are used as selection signal for the multiplexers and the last one, the MSB, for the adder. In fact, as it's possible to notice from the table \ref{mult:lut}, when the MSB is 1 the value to be added to the partial result is positive, negative otherwise. For this reason, the inputs to the multiplexers are only $\{0, A, 2A\}$. Since we need also to generate negative values, like ${-A, -2A}$, the MSB of the three bits is used as input for the adder. This signal, called \texttt{SUB\_SUMn} is used to define if the operation is a sum or a subtraction; if it is 1, a subtraction is performed;
	\item $N/2$ multiplexers that select only among $\{0, A, 2A\}$, since at each stage $A$ must be multiplied by 4, a shift by two is done starting from $A$ of the previous multiplexer;
	\item $N/2$ ripple carry adders, that allow to preform the partial sums. Since the final results will be on $NBIT \cdot 2+1$ bits, the adders in each level have been optimized in order to work only with the minimum bits needed. In fact, the adder at $i$ level, will generate the result on $NBIT + 2 \cdot i$ bits. As said before, an addition signal called \texttt{SUB\_SUMn} has been added in order to be able to perform the subtraction. The Ripple Carry Adder has been select for its simplicity and since the multiplication is a less common instruction, it was not worth to use a more sophisticated adder. This allowed to reduce the total area of the multiplier itself.
\end{enumerate}
The multiplexer implements the \textit{LUT} and at the same time the $A = A \cdot 4$. The two values from the two multiplexer are summed together via an adder, this implements the partial sum. The overall structure can be observed at \ref{fig:multiplier}.

\begin{figure}[ht]
	\centering
	\includegraphics[width=0.8\textwidth]{chapters/5_ExecuteStage/images/multiplier.pdf}
	\caption{Booth's multiplier on 32 bits}
	\label{fig:multiplier}
\end{figure}

	
\subsection{Logic Operands}
The basic and most simple implementation of a logic unit is based on single logic gates on $N$ bits whose outputs are muxed, in order to generate the correct output. The problem with this solution is that the number of input signals to the multiplexer is extremely high; this implementation does not only suffer from the point of view of the delay but, since each logic function is implemented with a specific gate, the total area is huge.\newline\newline
In order to overcome the problems highlighted before, a more compact implementation has been chosen: the T2 logic unit.

This logic unit allows to perform AND, NAND, OR, NOR, XOR and XNOR using only 5 NAND gates, on two levels, and 4 selection signals. The schematic is the one in figure \ref{fig:log_unit}.

\begin{figure}
	\centering
	\tikzstyle{branch}=[fill,shape=circle,minimum size=3pt,inner sep=0pt]
	\begin{tikzpicture}[label distance=2mm]
		\draw (0.92,-0.40) -- (1.09,-0.56);
		\draw (1.92,-0.40) -- (2.09,-0.56);
		% nodes
		\node (y1) at (1,0) {$R_1$};
		\node (y2) at (2,0) {$R_2$};
		\node[not gate US, draw, rotate=-90] at ($(y1)+(0.5,-1.5)$) (noty1) {};
		\node[not gate US, draw, rotate=-90] at ($(y2)+(0.5,-1.5)$) (noty2) {};
		
		% draw nodes to NOT
		\foreach \i in {1,2} {
			\path (y\i) -- coordinate (punt\i) (y\i |- noty\i.input);
			\draw (punt\i) node[branch] {} -| (noty\i.input);
		}
	
		\node (x1) at (0,-2.33) {$S_0$};
		\node (x2) at (0,-3.33) {$S_1$};
		\node (x3) at (0,-4.33) {$S_2$};
		\node (x4) at (0,-5.33) {$S_3$};
		
		\node[nand gate US, draw, logic gate inputs=nnn] at ($(y2)+(2,-2.5)$) (And1) {};
		\node[nand gate US, draw, logic gate inputs=nnn] at ($(And1)+(0,-1)$) (And2) {};
		\node[nand gate US, draw, logic gate inputs=nnn] at ($(And2)+(0,-1)$) (And3) {};
		\node[nand gate US, draw, logic gate inputs=nnn] at ($(And3)+(0,-1)$) (And4) {};
		\node[nand gate US, draw, logic gate inputs=nnnn, anchor=input 1] at ($(And1.output -| And2.output)+(2,-1.25)$) (Or1) {};
		

		% connect x_i to AND_i
		\foreach \i in {1,2,3,4} {
			\draw (x\i) -- (And\i.input 1);
		}
		
		% y1

		\draw (noty1 |- And1.input 2) node[branch] {} -- (And1.input 2);
		\draw (noty1 |- And2.input 2) node[branch] {} -- (And2.input 2);
		\draw (y1 |- And3.input 2) node[branch] {} -- (And3.input 2);
		\draw (y1) |- (And4.input 2);
		\draw (noty1) |- (And2.input 2);
		
		\draw (noty2 |- And1.input 3) node[branch] {} -- (And1.input 3);
		\draw (y2 |- And2.input 3) node[branch] {} -- (And2.input 3);
		\draw (noty2 |- And3.input 3) node[branch] {} -- (And3.input 3);
		\draw (y2) |- (And4.input 3);
		\draw (noty2) |- (And3.input 3);
		

		% AND
		\draw (And1.output) -- ([xshift=0.8cm]And1.output) |- (Or1.input 1);
		\draw (And2.output) -- ([xshift=0.6cm]And2.output) |- (Or1.input 2);
		\draw (And3.output) -- ([xshift=0.6cm]And3.output) |- (Or1.input 3);
		\draw (And4.output) -- ([xshift=0.8cm]And4.output) |- (Or1.input 4);
	
		
		% OR
		\draw (Or1.output) -- ([xshift=0.5cm]Or1.output) node[above] {$out$};
		
	\end{tikzpicture}  
	\caption{Logic unit}
	\label{fig:log_unit}
	\end{figure}

	In order to compute one of the logical instructions, the select signals are properly activated as follow:
	
	\[
	\begin{vmatrix}
		S_0 & S_1 & S_2 & S_3 & \text{operation}\\
		0 & 0 & 0 & 1 & AND \\
		1 & 1 & 1 & 0 & NAND \\
		0 & 1 & 1 & 1 & OR \\
		1 & 0 & 0 & 0 & NOR \\
		0 & 1 & 1 & 0 & XOR \\
		1 & 0 & 0 & 1 & NXOR \\
	\end{vmatrix}
	\]
	
	
	For example, in order to generate the AND logical operation, we have to select $S_3 = 1$, so that $out = R_1 \cdot R_2$; on the other hand, if we need NAND $S_0 = S_1 = S_2 = 1$ and $S_3 = 0$, so that $out = \overline{R_1} \cdot \overline{R_2} + \overline{R_1} \cdot R_2 + R_1 \cdot \overline{R_2} = \overline{R_1} \cdot \overline{R_2}$ that using the De Morgan law $out = \overline{R_1 \cdot R_2}$.
	This allows to obtain the best performances also because all paths work in parallel, compacting the area and the delay.

\subsection{Shifting}
The implemented shifter allows to perform shift right, logical/arithmetical shift left and left/right rotate using the full operand \texttt{A} on 32 bits and 6 bits from the second one \texttt{B} and three \textit{control signals}.
Differently from the T2 version, it uses and addition signal in order to be able to manage also the rotate instruction. Our implementation takes three inputs:
\begin{itemize}
	\itemsep0sp
	\item \texttt{A}: the operand to be shifted/rotated;
	\item \texttt{B}: only the 5 LSB [4,3,2,1,0] are used to select first the mask to be used and then the starting point from that mask;
	\item \texttt{SEL}: it encodes the operation type; the second bit is used to select among arithmetic and logic, the third bit is used to select the direction of the shift/rotate (left/right) and the first one is used only if the operation is a rotate. This is the encoding:
	\begin{center}
		\begin{tabular}{c|l}
			\texttt{SEL} & \textbf{Operation}\\
			\hline
			000 & Shift logic right \\
			001 & Shift logic left \\
			010 & Shift arith right \\
			011 & Shift arith left \\
			100 & Rotate right \\
			101 & Shift right \\
		\end{tabular}
	\end{center}
\end{itemize}

\begin{figure}[ht]
	\centering
	\includegraphics[width=0.2\textwidth]{chapters/5_ExecuteStage/images/Shifter.pdf}
	\caption{Blocks of the Shifter/Rotate Unit}
	\label{fig:shifter}
\end{figure}

 The unit perform the requested operation in three stages, sketched in figure \ref{fig:shifter}:
\begin{enumerate}
	\item The first consist in preparing 4 possible ``masks", each already shifted of {0, 8, 16, 32} left
	or right depending on the configuration. This allows to shift for all 32 bits. Basically it copies
	the input \texttt{A} into the 4 masks that will be used by the next stage. Being in 32 bits, the generated masks are in $32+8=40$ bits. The only different between this implementation and the T2 one, is that, in case of rotate, the additional 8 bits of the masks are filled with the corresponding 8 bits that are going ``out" during the rotation.
	
	\item The second level perform a coarse grain shift, that is basically consist on selecting one mask
	among the 4 possible masks generated in the previous stage. This selection is done by using the bits {4, 3} of \texttt{B}.
	\item The third level, using the bits {2, 1, 0} of \texttt{B} and the selected mask, preform a fine grain refinement. The 3 bits allows to select the starting index from the mask, in fact it allows to select among 8 positions.
\end{enumerate}



\begin{figure}[h] 
	\label{ fig7} 
	\begin{minipage}[b]{0.5\linewidth}
		\centering
		\includegraphics[width=.78\linewidth]{chapters/5_ExecuteStage/images/left_shift.pdf} 
		\caption{Masks for left shift} 
		\vspace{4ex}
	\end{minipage}%%
	\begin{minipage}[b]{0.5\linewidth}
		\centering
		\includegraphics[width=.78\linewidth]{chapters/5_ExecuteStage/images/right_shift.pdf}
		\caption{Masks for right shift} 
		\vspace{4ex}
	\end{minipage} 
	\begin{minipage}[b]{0.5\linewidth}
		\centering
		\includegraphics[width=.78\linewidth]{chapters/5_ExecuteStage/images/left_rotate.pdf} 
		\caption{Masks for left rotate} 
		\vspace{4ex}
	\end{minipage}%% 
	\begin{minipage}[b]{0.5\linewidth}
		\centering
		\includegraphics[width=.78\linewidth]{chapters/5_ExecuteStage/images/right_rotate.pdf} 
		\caption{Masks for right rotate} 
		\vspace{4ex}
	\end{minipage} 
	\caption{Masks for shift unit on 32 bits} 
\end{figure}
\begin{mybox}
	\textbf{Examples}
	\newline
	For example, if we need to perform a left left of 9 bits \texttt{A}, where \texttt{A=18}, the corresponding \texttt{B} value will be 1001; this means that the second masks will be taken and the output result will from the bit at position $40-1=39$ to the one at $39-32=7$ included.
	\begin{center}
		MASK 2: 0$\underbrace{\textbf{0000000 00000000 00000000 00010010 0}}_{\text{shifted \texttt{A}}}$0000000
	\end{center}
	
	On the other hand, if we need to perform a right shift the masks are generated in the opposite way, so the zeros are put in the MSB of the mask, shifted by 0, 8 ... positions. In this case we need also to distinguish between the an arithmetic and a logic shift; in the first case, instead of filling the ``empty" bits with zero, the operand sign is used. For example, if we want to shift \texttt{A=-18} of \texttt{B=3} bits, the first mask is used: 
	\begin{center}
		MASK 1: 11111$\underbrace{\textbf{111 11111111 11111111 11111111 11101}}_{\text{shifted \texttt{A}}}$110
	\end{center}
	
	In the last case, let's suppose to rotate right \texttt{A=1255} (=10011100111) by 5 position:
	\begin{center}
		MASK 1: 11$\underbrace{\textbf{100111 00000000 00000000 0000100 111}}_{\text{rotated \texttt{A}}}$00111
	\end{center}
	As you can see, in case of MASK 1 for the right rotation, the 8 LSB of \texttt{A} are copied into the 8 MSB of the mask.
\end{mybox}
	

\section{Set-Like Operations unit}
- setcmp